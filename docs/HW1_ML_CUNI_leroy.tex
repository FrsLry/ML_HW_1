%%%%%%%%%%%%%%%%%%%%%%%%%%%%%%%%%%%%%%%%%%%%%%%%

% Specify the command that you want into the header of the
% index.md file

%%%%%%%%%%%%%%%%%%%%%%%%%%%%%%%%%%%%%%%%%%%%%%%%

% Options for packages loaded elsewhere
\PassOptionsToPackage{unicode}{hyperref}
\PassOptionsToPackage{hyphens}{url}
\PassOptionsToPackage{dvipsnames,svgnames*,x11names*}{xcolor}
%
\documentclass[
  12pt,
  oneside]{report}
%%\usepackage{lmodern}
%
% Set line spacing
\usepackage{setspace}
\setstretch{1.5}

\usepackage{amssymb,amsmath}
\usepackage{ifxetex,ifluatex}
\ifnum 0\ifxetex 1\fi\ifluatex 1\fi=0 % if pdftex
  \usepackage[T1]{fontenc}
  \usepackage[utf8]{inputenc}
  \usepackage{textcomp} % provide euro and other symbols
\else % if luatex or xetex
  \usepackage{unicode-math}
  \defaultfontfeatures{Scale=MatchLowercase}
  \defaultfontfeatures[\rmfamily]{Ligatures=TeX,Scale=1}
\fi
% Use upquote if available, for straight quotes in verbatim environments
\IfFileExists{upquote.sty}{\usepackage{upquote}}{}
\IfFileExists{microtype.sty}{% use microtype if available
  \usepackage[]{microtype}
  \UseMicrotypeSet[protrusion]{basicmath} % disable protrusion for tt fonts
}{}
\makeatletter
\@ifundefined{KOMAClassName}{% if non-KOMA class
  \IfFileExists{parskip.sty}{%
    \usepackage{parskip}
  }{% else
    \setlength{\parindent}{0pt}
    \setlength{\parskip}{6pt plus 2pt minus 1pt}}
}{% if KOMA class
  \KOMAoptions{parskip=half}}
\makeatother
\usepackage{xcolor}
\IfFileExists{xurl.sty}{\usepackage{xurl}}{} % add URL line breaks if available
\IfFileExists{bookmark.sty}{\usepackage{bookmark}}{\usepackage{hyperref}}
\hypersetup{
  pdfauthor={François Leroy, PhD student at CZU},
  colorlinks=true,
  linkcolor=Maroon,
  filecolor=Maroon,
  citecolor=Blue,
  urlcolor=Blue,
  pdfcreator={LaTeX via pandoc}}
\urlstyle{same} % disable monospaced font for URLs

%% Package geometry
\usepackage[left = 2cm,right = 2cm,top = 2cm,bottom = 2cm]{geometry}
\usepackage{pdflscape}


\usepackage{color}
\usepackage{fancyvrb}
\newcommand{\VerbBar}{|}
\newcommand{\VERB}{\Verb[commandchars=\\\{\}]}
\DefineVerbatimEnvironment{Highlighting}{Verbatim}{commandchars=\\\{\}}
% Add ',fontsize=\small' for more characters per line
\usepackage{framed}
\definecolor{shadecolor}{RGB}{248,248,248}
\newenvironment{Shaded}{\begin{snugshade}}{\end{snugshade}}
\newcommand{\AlertTok}[1]{\textcolor[rgb]{0.94,0.16,0.16}{#1}}
\newcommand{\AnnotationTok}[1]{\textcolor[rgb]{0.56,0.35,0.01}{\textbf{\textit{#1}}}}
\newcommand{\AttributeTok}[1]{\textcolor[rgb]{0.77,0.63,0.00}{#1}}
\newcommand{\BaseNTok}[1]{\textcolor[rgb]{0.00,0.00,0.81}{#1}}
\newcommand{\BuiltInTok}[1]{#1}
\newcommand{\CharTok}[1]{\textcolor[rgb]{0.31,0.60,0.02}{#1}}
\newcommand{\CommentTok}[1]{\textcolor[rgb]{0.56,0.35,0.01}{\textit{#1}}}
\newcommand{\CommentVarTok}[1]{\textcolor[rgb]{0.56,0.35,0.01}{\textbf{\textit{#1}}}}
\newcommand{\ConstantTok}[1]{\textcolor[rgb]{0.00,0.00,0.00}{#1}}
\newcommand{\ControlFlowTok}[1]{\textcolor[rgb]{0.13,0.29,0.53}{\textbf{#1}}}
\newcommand{\DataTypeTok}[1]{\textcolor[rgb]{0.13,0.29,0.53}{#1}}
\newcommand{\DecValTok}[1]{\textcolor[rgb]{0.00,0.00,0.81}{#1}}
\newcommand{\DocumentationTok}[1]{\textcolor[rgb]{0.56,0.35,0.01}{\textbf{\textit{#1}}}}
\newcommand{\ErrorTok}[1]{\textcolor[rgb]{0.64,0.00,0.00}{\textbf{#1}}}
\newcommand{\ExtensionTok}[1]{#1}
\newcommand{\FloatTok}[1]{\textcolor[rgb]{0.00,0.00,0.81}{#1}}
\newcommand{\FunctionTok}[1]{\textcolor[rgb]{0.00,0.00,0.00}{#1}}
\newcommand{\ImportTok}[1]{#1}
\newcommand{\InformationTok}[1]{\textcolor[rgb]{0.56,0.35,0.01}{\textbf{\textit{#1}}}}
\newcommand{\KeywordTok}[1]{\textcolor[rgb]{0.13,0.29,0.53}{\textbf{#1}}}
\newcommand{\NormalTok}[1]{#1}
\newcommand{\OperatorTok}[1]{\textcolor[rgb]{0.81,0.36,0.00}{\textbf{#1}}}
\newcommand{\OtherTok}[1]{\textcolor[rgb]{0.56,0.35,0.01}{#1}}
\newcommand{\PreprocessorTok}[1]{\textcolor[rgb]{0.56,0.35,0.01}{\textit{#1}}}
\newcommand{\RegionMarkerTok}[1]{#1}
\newcommand{\SpecialCharTok}[1]{\textcolor[rgb]{0.00,0.00,0.00}{#1}}
\newcommand{\SpecialStringTok}[1]{\textcolor[rgb]{0.31,0.60,0.02}{#1}}
\newcommand{\StringTok}[1]{\textcolor[rgb]{0.31,0.60,0.02}{#1}}
\newcommand{\VariableTok}[1]{\textcolor[rgb]{0.00,0.00,0.00}{#1}}
\newcommand{\VerbatimStringTok}[1]{\textcolor[rgb]{0.31,0.60,0.02}{#1}}
\newcommand{\WarningTok}[1]{\textcolor[rgb]{0.56,0.35,0.01}{\textbf{\textit{#1}}}}
\usepackage{longtable,booktabs}
% Correct order of tables after \paragraph or \subparagraph
\usepackage{etoolbox}
\makeatletter
\patchcmd\longtable{\par}{\if@noskipsec\mbox{}\fi\par}{}{}
\makeatother
% Allow footnotes in longtable head/foot
\IfFileExists{footnotehyper.sty}{\usepackage{footnotehyper}}{\usepackage{footnote}}
\makesavenoteenv{longtable}
\usepackage{graphicx}
\makeatletter
\def\maxwidth{\ifdim\Gin@nat@width>\linewidth\linewidth\else\Gin@nat@width\fi}
\def\maxheight{\ifdim\Gin@nat@height>\textheight\textheight\else\Gin@nat@height\fi}
\makeatother
% Scale images if necessary, so that they will not overflow the page
% margins by default, and it is still possible to overwrite the defaults
% using explicit options in \includegraphics[width, height, ...]{}
\setkeys{Gin}{width=\maxwidth,height=\maxheight,keepaspectratio}
% Set default figure placement to htbp
\makeatletter
\def\fps@figure{htbp}
\makeatother
\setlength{\emergencystretch}{3em} % prevent overfull lines
\providecommand{\tightlist}{%
  \setlength{\itemsep}{0pt}\setlength{\parskip}{0pt}}
\setcounter{secnumdepth}{5}
%%% Complete the preamble of the LaTeX template
%%%------------------------------------------------------------------------------

%% Bug de bookdown: ne traite plus la déclaration "otherlangs" dans le préambule
% Pour charger les langues, écriture ici en dur du produit de bookdown
% Corrigé le 22/11/2019. A retester régulièrement: supprimer ces lignes si la compilation fonctionne sans elles.
\usepackage{polyglossia}
  \setmainlanguage[variant=american]{english}
  \setotherlanguage[]{french}
% Bug persistant le 28/02/2020

% Advised with polyglossia and babel
\usepackage{csquotes}

% Environnement "Essentiel" en début de chapitre
\usepackage[tikz]{bclogo}
\newenvironment{Essentiel}
  {\begin{bclogo}[logo=\bctrombone, noborder=true, couleur=lightgray!50]{L'essentiel}\parindent0pt}
  {\end{bclogo}}

%% Package fontspec
\usepackage{fontspec}
\setmainfont{calibri}[
  Path           = ./fonts/,
  Extension      = .ttf,
  BoldFont       = calibrib,
  ItalicFont     = calibrili,
  BoldItalicFont = calibriz]

% Rename chapters
% Below, scrpit to prevent the "chapter n" and the space use for it to
% be displayed
\usepackage{titlesec}
\titleformat{\chapter}   
{\Huge}{\thechapter{. }}{0pt}{\Huge}
%{\thechapter{. }}
\titlespacing*{\chapter}{0pt}{-50pt}{10pt}
% -50 is to up the title and 10 is the space with the text below
\ifluatex
  \usepackage{selnolig}  % disable illegal ligatures
\fi

\title{Introduction to Machine Learning\\
(NPFL054)}
\usepackage{etoolbox}
\makeatletter
\providecommand{\subtitle}[1]{% add subtitle to \maketitle
  \apptocmd{\@title}{\par {\large #1 \par}}{}{}
}
\makeatother
\subtitle{Homework 1}
\author{François Leroy, PhD student at CZU}
\date{2021-04-17}

% to include pdf
\usepackage{pdfpages}



%%%%%%%%%%%%%%%%%%%%%%%%%%%%%%%%%%%%%%%%%%%%%%%%%%%%%%%%%%%%%
% Start of the documents
\begin{document}
\maketitle


% Roman numbering for content before toc and toc itself
\cleardoublepage 
\pagenumbering{roman}

{
\hypersetup{linkcolor=}
\setcounter{tocdepth}{1}
\tableofcontents
\newpage
}
\vspace{50mm}
\setstretch{1.5}


% Start the arabic numbering at the 1st chapter
\cleardoublepage 
\pagenumbering{arabic}


% The mind, the...
\hypertarget{q1}{%
\chapter{Multiple linear regression}\label{q1}}

\hypertarget{section}{%
\section{}\label{section}}

\begin{Shaded}
\begin{Highlighting}[]
\KeywordTok{library}\NormalTok{(ISLR)}
\KeywordTok{library}\NormalTok{(tidyverse)}
\end{Highlighting}
\end{Shaded}

\begin{Shaded}
\begin{Highlighting}[]
\CommentTok{# Perform the multiple linear regression}
\NormalTok{lm <-}\StringTok{ }
\StringTok{  }\KeywordTok{lm}\NormalTok{(mpg }\OperatorTok{~}\StringTok{ }\NormalTok{., }\DataTypeTok{data =} \KeywordTok{subset}\NormalTok{(Auto, }\DataTypeTok{select =} \OperatorTok{-}\NormalTok{name))}
\CommentTok{# Print the output}
\KeywordTok{summary}\NormalTok{(lm)}
\end{Highlighting}
\end{Shaded}

\begin{verbatim}
## 
## Call:
## lm(formula = mpg ~ ., data = subset(Auto, select = -name))
## 
## Residuals:
##     Min      1Q  Median      3Q     Max 
## -9.5903 -2.1565 -0.1169  1.8690 13.0604 
## 
## Coefficients:
##                Estimate Std. Error t value Pr(>|t|)    
## (Intercept)  -17.218435   4.644294  -3.707  0.00024 ***
## cylinders     -0.493376   0.323282  -1.526  0.12780    
## displacement   0.019896   0.007515   2.647  0.00844 ** 
## horsepower    -0.016951   0.013787  -1.230  0.21963    
## weight        -0.006474   0.000652  -9.929  < 2e-16 ***
## acceleration   0.080576   0.098845   0.815  0.41548    
## year           0.750773   0.050973  14.729  < 2e-16 ***
## origin         1.426141   0.278136   5.127 4.67e-07 ***
## ---
## Signif. codes:  0 '***' 0.001 '**' 0.01 '*' 0.05 '.' 0.1 ' ' 1
## 
## Residual standard error: 3.328 on 384 degrees of freedom
## Multiple R-squared:  0.8215, Adjusted R-squared:  0.8182 
## F-statistic: 252.4 on 7 and 384 DF,  p-value: < 2.2e-16
\end{verbatim}

First of all, the adjusted \(R^2 = 0.82\), which means that 82\% of the variance of the data is explained by this models. This is a very trustful model.

Here, I will talk about only about the covariates that have a significant influence (\emph{i.e.} \(p-value \leq 0.05\)) on the \texttt{mpg} variable (\emph{i.e.} rows with an asterix such as \texttt{displacement}, \texttt{weight}, \texttt{year} and \texttt{origin}):

\begin{itemize}
\item
  The miles per galon unit (\emph{i.e.} \texttt{mpg}) expresses the fuel economy of a vehicle. Thus, when the coefficient of the \texttt{lm} is negative, it means that the vehicle will tend to go less further with a unit of fuel. Here, this is the case for the \texttt{weight} variable: a heavier vehicle will consume more fuel than a lighter one.
\item
  The other significant relationships with the \texttt{displacement}, \texttt{year} and \texttt{origin} are positive which means that a more recent car, with a higher displacement volume and with a higher origin will tend to consume less fuel.
\end{itemize}

\newpage

\hypertarget{section-1}{%
\section{}\label{section-1}}

\begin{Shaded}
\begin{Highlighting}[]
\CommentTok{## Perform the 5 polynomial simpple linear regression}
\ControlFlowTok{for}\NormalTok{ (i }\ControlFlowTok{in} \DecValTok{1}\OperatorTok{:}\DecValTok{5}\NormalTok{)\{}
  \KeywordTok{assign}\NormalTok{(}\KeywordTok{paste0}\NormalTok{(}\StringTok{"fit"}\NormalTok{, i),}
         \KeywordTok{lm}\NormalTok{(mpg }\OperatorTok{~}\StringTok{ }\KeywordTok{poly}\NormalTok{(acceleration, i), }\DataTypeTok{data =} \KeywordTok{subset}\NormalTok{(Auto, }\DataTypeTok{select =} \OperatorTok{-}\NormalTok{name)))}
\NormalTok{\}}
\CommentTok{## Plot them on a single plot}
\CommentTok{#### First merge the predicted values of mpg with the acceleration}
\NormalTok{Auto }\OperatorTok\StringTok{ }
\StringTok{  }\KeywordTok{select}\NormalTok{(acceleration) }\OperatorTok\StringTok{ }
\StringTok{  }\KeywordTok{cbind}\NormalTok{(}\DataTypeTok{poly1 =}\NormalTok{ fit1}\OperatorTok{$}\NormalTok{fitted.values,}
        \DataTypeTok{poly2 =}\NormalTok{ fit2}\OperatorTok{$}\NormalTok{fitted.values,}
        \DataTypeTok{poly3 =}\NormalTok{ fit3}\OperatorTok{$}\NormalTok{fitted.values,}
        \DataTypeTok{poly4 =}\NormalTok{ fit4}\OperatorTok{$}\NormalTok{fitted.values,}
        \DataTypeTok{poly5 =}\NormalTok{ fit5}\OperatorTok{$}\NormalTok{fitted.values) }\OperatorTok
\CommentTok{##### Then format the data for ggplot  }
\StringTok{  }\KeywordTok{pivot_longer}\NormalTok{(}\DataTypeTok{cols =}\NormalTok{ poly1}\OperatorTok{:}\NormalTok{poly5,}
               \DataTypeTok{names_to =} \StringTok{"poly"}\NormalTok{,}
               \DataTypeTok{values_to =} \StringTok{"mpg"}\NormalTok{) }\OperatorTok\StringTok{ }
\StringTok{  }\KeywordTok{mutate}\NormalTok{(}\DataTypeTok{rsq =} \KeywordTok{case_when}\NormalTok{(}
\NormalTok{    poly }\OperatorTok{==}\StringTok{ "poly1"} \OperatorTok{~}\StringTok{ }\KeywordTok{round}\NormalTok{(}\KeywordTok{summary}\NormalTok{(fit1)}\OperatorTok{$}\NormalTok{adj.r.squared, }\DataTypeTok{digits =} \DecValTok{2}\NormalTok{),}
\NormalTok{    poly }\OperatorTok{==}\StringTok{ "poly2"} \OperatorTok{~}\StringTok{ }\KeywordTok{round}\NormalTok{(}\KeywordTok{summary}\NormalTok{(fit2)}\OperatorTok{$}\NormalTok{adj.r.squared, }\DataTypeTok{digits =} \DecValTok{2}\NormalTok{),}
\NormalTok{    poly }\OperatorTok{==}\StringTok{ "poly3"} \OperatorTok{~}\StringTok{ }\KeywordTok{round}\NormalTok{(}\KeywordTok{summary}\NormalTok{(fit3)}\OperatorTok{$}\NormalTok{adj.r.squared, }\DataTypeTok{digits =} \DecValTok{2}\NormalTok{),}
\NormalTok{    poly }\OperatorTok{==}\StringTok{ "poly4"} \OperatorTok{~}\StringTok{ }\KeywordTok{round}\NormalTok{(}\KeywordTok{summary}\NormalTok{(fit4)}\OperatorTok{$}\NormalTok{adj.r.squared, }\DataTypeTok{digits =} \DecValTok{2}\NormalTok{),}
\NormalTok{    poly }\OperatorTok{==}\StringTok{ "poly5"} \OperatorTok{~}\StringTok{ }\KeywordTok{round}\NormalTok{(}\KeywordTok{summary}\NormalTok{(fit5)}\OperatorTok{$}\NormalTok{adj.r.squared, }\DataTypeTok{digits =} \DecValTok{2}\NormalTok{)}
\NormalTok{  )) }\OperatorTok\StringTok{ }
\StringTok{  }\KeywordTok{unite}\NormalTok{(poly, }\KeywordTok{c}\NormalTok{(}\StringTok{"poly"}\NormalTok{, }\StringTok{"rsq"}\NormalTok{), }\DataTypeTok{sep =} \StringTok{", adj.R² = "}\NormalTok{) }\OperatorTok\StringTok{ }
\CommentTok{#### Now plot it}
\StringTok{  }\KeywordTok{ggplot}\NormalTok{()}\OperatorTok{+}
\StringTok{  }\KeywordTok{geom_point}\NormalTok{(}\KeywordTok{aes}\NormalTok{(acceleration, mpg), }\DataTypeTok{data =} \KeywordTok{subset}\NormalTok{(Auto, }\DataTypeTok{select =} \OperatorTok{-}\NormalTok{name))}\OperatorTok{+}
\StringTok{  }\KeywordTok{geom_line}\NormalTok{(}\KeywordTok{aes}\NormalTok{(acceleration, mpg, }\DataTypeTok{color =}\NormalTok{ poly), }\DataTypeTok{size =} \FloatTok{1.2}\NormalTok{)}\OperatorTok{+}
\StringTok{  }\KeywordTok{theme_classic}\NormalTok{()}
\end{Highlighting}
\end{Shaded}

\includegraphics{HW1_ML_CUNI_leroy_files/figure-latex/unnamed-chunk-4-1.pdf}


\singlespacing % reset the spacing of the bibliography style
\end{document}
